\documentclass[12pt, a4paper]{article}

\usepackage[utf8]{inputenc}
\usepackage[russian]{babel}
\parindent 0pt
\parskip 8pt
\usepackage{amsmath}
\usepackage{amssymb}
\usepackage{array}
\usepackage{floatrow}
\usepackage{float}
\usepackage[left=2.3cm, right=2.3cm, top=2.7cm, bottom=2.7cm, bindingoffset=0cm]{geometry}
\usepackage{hyperref}
\usepackage{graphicx}
\usepackage{multicol}
\usepackage{listings}
\usepackage{fancyhdr} 
\usepackage{extramarks}
\usepackage[usenames,dvipsnames]{color}
\usepackage{titlesec}
\usepackage{tikz}
\usepackage[T2A]{fontenc} 
\definecolor{grey}{RGB}{128,128,128}

\begin{document}


\section{Графики}

Нормальное распределение:

\includegraphics[scale=0.6]{n.png}

Равномерное распределение:

\includegraphics[scale=0.6]{u.png}

\newpage

\section{Критерии Колмогорова и Смирнова}

Нормальное распределение:
\begin{center}
    \begin{tabular}{|c|c|c|}
         \hline \textbf{n} & \textbf{Критерий Колмогорова} & \textbf{Критерий Смирнова} \\
         \hline 10^4 & 1.014 & 0.272 \\
         \hline 10^6 & 0.867 & 0.101 \\
         \hline
    \end{tabular}
\end{center}


Равномерное распределение:
\begin{center}
    \begin{tabular}{|c|c|c|}
         \hline \textbf{n} & \textbf{Критерий Колмогорова} & \textbf{Критерий Смирнова} \\
         \hline 10^4 & 1.195 & 0.313 \\
         \hline 10^6 & 0.529 & 0.022 \\
         \hline
    \end{tabular}
\end{center}

Значения Критерия Колмогорова во всех случаях $< 1.36$, следовательно, гипотеза согласуется с эксперементальными данными Колмогрова-Смирнова

Значения Критерия Смиронова во всех случаях $< 0.46$, следовательно, гипотеза согласуется с эксперементальными данными Мизеса-Смирнова

\newpage

\section{Ошибки I рода}

\subsection{Нормальное распределение}

$n = 10^4, m = 10^2$:
\begin{center}
    \begin{tabular}{|c|c|c|}
         \hline \boldmath$\gamma$ & \textbf{Критерий Колмогорова} & \textbf{Критерий Смирнова} \\
         \hline 0.900 & 0.090 & 0.090 \\
         \hline 0.950 & 0.040 & 0.050 \\
         \hline 0.990 & 0.010 & 0.000 \\
         \hline
    \end{tabular}
\end{center}

$n = 10^6, m = 10^2$:

\begin{center}
    \begin{tabular}{|c|c|c|}
         \hline \boldmath$\gamma$ & \textbf{Критерий Колмогорова} & \textbf{Критерий Смирнова} \\
         \hline 0.900 & 0.090 & 0.090 \\
         \hline 0.950 & 0.080 & 0.070  \\
         \hline 0.990 & 0.010 & 0.010 \\
         \hline
    \end{tabular}
\end{center}

\subsection{Равномерное распределение}

$n = 10^4, m = 10^2$:
\begin{center}
    \begin{tabular}{|c|c|c|}
         \hline \boldmath$\gamma$ & \textbf{Критерий Колмогорова} & \textbf{Критерий Смирнова} \\
         \hline 0.900&0.090&0.120 \\
         \hline 0.950&0.100&0.040 \\
         \hline 0.990&0.000&0.000 \\
         \hline
    \end{tabular}
\end{center}

$n = 10^6, m = 10^2$:

\begin{center}
    \begin{tabular}{|c|c|c|}
         \hline \boldmath$\gamma$ & \textbf{Критерий Колмогорова} & \textbf{Критерий Смирнова} \\
         \hline 0.900&0.090&0.100 \\
         \hline 0.950&0.020&0.030  \\
         \hline 0.990&0.010&0.020 \\
         \hline
    \end{tabular}
\end{center}

\newpage
\section{Ошибки II рода}


\subsection{Нормальное распределение}

$n = 10^4, m = 10^2$:
\begin{center}
    \begin{tabular}{|c|c|c|}
         \hline \boldmath$\gamma$ & \textbf{Критерий Колмогорова} & \textbf{Критерий Смирнова} \\
         \hline 0.900 & 0.370 & 0.340 \\
         \hline 0.950 & 0.670 & 0.570 \\
         \hline 0.990 & 0.870 & 0.870 \\
         \hline
    \end{tabular}
\end{center}

$n = 10^6, m = 10^2$:

\begin{center}
    \begin{tabular}{|c|c|c|}
         \hline \boldmath$\gamma$ & \textbf{Критерий Колмогорова} & \textbf{Критерий Смирнова} \\
         \hline 0.900 & 0.000 & 0.000 \\
         \hline 0.950 & 0.000 & 0.000  \\
         \hline 0.990 & 0.000 & 0.000 \\
         \hline
    \end{tabular}
\end{center}

\subsection{Равномерное распределение}

$n = 10^4, m = 10^2$:
\begin{center}
    \begin{tabular}{|c|c|c|}
         \hline \boldmath$\gamma$ & \textbf{Критерий Колмогорова} & \textbf{Критерий Смирнова} \\
         \hline 0.900 & 0.000 & 0.000 \\
         \hline 0.950 & 0.000 & 0.010  \\
         \hline 0.990 & 0.000 & 0.090 \\
         \hline
    \end{tabular}
\end{center}

$n = 10^6, m = 10^2$:

\begin{center}
    \begin{tabular}{|c|c|c|}
         \hline \boldmath$\gamma$ & \textbf{Критерий Колмогорова} & \textbf{Критерий Смирнова} \\
         \hline 0.900 & 0.000 & 0.000 \\
         \hline 0.950 & 0.000 & 0.000  \\
         \hline 0.990 & 0.000 & 0.000 \\
         \hline
    \end{tabular}
\end{center}

\newpage

\section{Выводы}

\begin{itemize}
    \item Функция распределения лежит в доверительной полосе.
    \item  Для нормального и равномерного распределения гипотеза выполяется - полученные     значения критериев Колмогорова и Смирнова меньше значений квантилей соответственно.
    \item  Вероятность ошибки I рода стремится к $1 - \gamma$
    \item  Вероятность ошибки II рода стремится к $0$ при увеличении $n$
\end{itemize}


\end{document}