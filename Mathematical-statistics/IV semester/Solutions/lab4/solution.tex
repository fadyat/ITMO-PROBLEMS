\documentclass[12pt, a4paper]{article}

\usepackage[utf8]{inputenc}
\usepackage[russian]{babel}
\parindent 0pt
\parskip 8pt
\usepackage{amsmath}
\usepackage{amssymb}
\usepackage{array}
\usepackage{floatrow}
\usepackage{float}
\usepackage[left=2.3cm, right=2.3cm, top=2.7cm, bottom=2.7cm, bindingoffset=0cm]{geometry}
\usepackage{hyperref}
\usepackage{graphicx}
\usepackage{multicol}
\usepackage{listings}
\usepackage{fancyhdr} 
\usepackage{extramarks}
\usepackage[usenames,dvipsnames]{color}
\usepackage{titlesec}
\usepackage{tikz}
\usepackage[T2A]{fontenc} 
\definecolor{grey}{RGB}{128,128,128}

\begin{document}

\section{Построение гистограмм}

Нормальное распределение $X \sim N(0, 3)$

\includegraphics[scale=0.9]{n.png}

Равномерное распределение $X \sim U(-3, 9)$

\includegraphics[scale=0.9]{u.png}

Графики гистограмм схожи с графиками соответствующих им плотностей распределения. С увеличением выборки это сходство будет увеличиваться

\newpage

\section{$\chi^2$-критерий Пирсона}


Проверим гипотезу о нормальном и равномерном распределении генеральной совокупности:

\subsection{Нормальное распределение}

$\mu = 0 \qquad \sigma = 3 \qquad \Delta_1 = 0 \qquad \Delta_2 = 0$

\begin{center}
    \begin{tabular}{|c|c|}
         \hline $\gamma$ & Acceptence \\
         \hline 0.9 & 1 \\
         \hline 0.95 & 1 \\
         \hline 0.99 & 1 \\
         \hline
    \end{tabular}
\end{center}

\subsection{Равномерное распределение}

$a = -3 \qquad b = 9 \qquad \Delta_1 = 0 \qquad \Delta_2 = 0$
\begin{center}
    \begin{tabular}{|c|c|}
         \hline $\gamma$ & Acceptence \\
         \hline 0.9 & 1 \\
         \hline 0.95 & 1 \\
         \hline 0.99 & 1 \\
         \hline
    \end{tabular}
\end{center}

Во всех случаях гипотеза подтверждается

\section{Ошибка I рода}

Число экспериментов = 100

\subsection{Нормальное распределение}

$\mu = 0 \qquad \sigma = 3 \qquad \Delta_1 = 0 \qquad \Delta_2 = 0$

\begin{center}
    \begin{tabular}{|c|c|}
         \hline $\gamma$ & error\_probability \\
         \hline 0.9 & 0.11 \\
         \hline 0.95 & 0.05 \\
         \hline 0.99 & 0.02 \\
         \hline
    \end{tabular}
\end{center}

\subsection{Равномерное распределение}

$a = -3 \qquad b = 9 \qquad \Delta_1 = 0 \qquad \Delta_2 = 0$

\begin{center}
    \begin{tabular}{|c|c|}
         \hline $\gamma$ & error\_probability \\
         \hline 0.9 & 0.09 \\
         \hline 0.95 & 0.06 \\
         \hline 0.99 & 0.01 \\
         \hline
    \end{tabular}
\end{center}

\section{Ошибка II рода}


Число экспериментов = 100

\subsection{Нормальное распределение}

\begin{tabular}{|c|c|c|c|}
     \hline $\gamma$ & error\_probability & \Delta_1 & \Delta_2 \\
     \hline 0.9 & 0.44 & 0 & 0.01 \\
     \hline 0.95 & 0 & 0 & 0.05 \\
     \hline 0.99 & 0 & 0 & 0.1 \\
     \hline
\end{tabular}

\subsection{Равномерное распределение}

\begin{tabular}{|c|c|c|c|}
     \hline $\gamma$ & error\_probability & \Delta_1 & \Delta_2 \\
     \hline 0.9 & 0.65 & 0 & 0.01 \\
     \hline 0.95 & 0 & 0 & 0.05 \\
     \hline 0.99 & 0 & 0 & 0.1 \\
     \hline
\end{tabular}

\section{Выводы}


\begin{itemize}
    \item Вероятность ошибки I рода стремится к $1 - \gamma$
    \item При увеличении сдвига, вероятность ошибки II рода уменьшается
\end{itemize}

\end{document}