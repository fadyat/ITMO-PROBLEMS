\documentclass[12pt, a4paper]{article}

\usepackage[utf8]{inputenc}
\usepackage[russian]{babel}
\parindent 0pt
\parskip 8pt
\usepackage{amsmath}
\usepackage{amssymb}
\usepackage{array}
\usepackage{floatrow}
\usepackage{float}
\usepackage[left=2.3cm, right=2.3cm, top=2.7cm, bottom=2.7cm, bindingoffset=0cm]{geometry}
\usepackage{hyperref}
\usepackage{graphicx}
\usepackage{multicol}
\usepackage{listings}
\usepackage{fancyhdr} 
\usepackage{extramarks}
\usepackage[usenames,dvipsnames]{color}
\usepackage{titlesec}
\usepackage{tikz}
\usepackage[T2A]{fontenc} 
\definecolor{grey}{RGB}{128,128,128}
\everymath{\displaystyle}

\begin{document}
\section{Normal distribution(1, 2)}

\begin{itemize}
    \item $n = 100, m = 100$

    \begin{tabular}{|c|c|c|c|}
        \hline & $mean(x)$ & $median(x)$ & $\frac{x_1 + x_n}{2}$ \\
        \hline $\sigma_t$ & 0.200 & 0.251 & 0.589 \\
        \hline $\sigma_p$ & 0.193 & 0.258 & 0.594 \\
        \hline $\Delta$ & 0.007 & 0.007 & 0.004 \\
        \hline
    \end{tabular}
    
    \item $n = 10000, m = 100$
    
    \begin{tabular}{|c|c|c|c|}
        \hline & $mean(x)$ & $median(x)$ & $\frac{x_1 + x_n}{2}$ \\
        \hline $\sigma_t$ & 0.020 & 0.025 & 0.417 \\
        \hline $\sigma_p$ & 0.021 & 0.026 & 0.398 \\
        \hline $\Delta$ & 0.001 & 0.001 & 0.019 \\
        \hline
    \end{tabular}
    
\end{itemize}

\newpage
\section{Uniform distribution(1, 5)}

\begin{itemize}
    \item $n = 100, m = 100$

    \begin{tabular}{|c|c|c|c|}
        \hline & $mean(x)$ & $median(x)$ & $\frac{x_1 + x_n}{2}$ \\
        \hline $\sigma_t$ & 0.115 & 0.200 & 0.028 \\
        \hline $\sigma_p$ & 0.120 & 0.214 & 0.028 \\
        \hline $\Delta$ & 0.005 & 0.014 & 0.000 \\
        \hline
    \end{tabular}
    
    \item $n = 10000, m = 100$
    
    \begin{tabular}{|c|c|c|c|}
        \hline & $mean(x)$ & $median(x)$ & $\frac{x_1 + x_n}{2}$ \\
        \hline $\sigma_t$ & 0.012 & 0.020 & 0.000 \\
        \hline $\sigma_p$ & 0.010 & 0.018 & 0.000 \\
        \hline $\Delta$ & 0.001 & 0.002 & 0.000 \\
        \hline
    \end{tabular}
    
\end{itemize}

\newpage
\section{Laplace distribution(5, 6)}

\begin{itemize}
    \item $n = 100, m = 100$

    \begin{tabular}{|c|c|c|c|}
        \hline & $mean(x)$ & $median(x)$ & $\frac{x_1 + x_n}{2}$ \\
        \hline $\sigma_t$ & 0.849 & 0.600 & 5.692 \\
        \hline $\sigma_p$ & 0.683 & 0.503 & 5.438 \\
        \hline $\Delta$ & 0.166 & 0.097 & 0.255 \\
        \hline
    \end{tabular}
    
    \item $n = 10000, m = 100$
    
    \begin{tabular}{|c|c|c|c|}
        \hline & $mean(x)$ & $median(x)$ & $\frac{x_1 + x_n}{2}$ \\
        \hline $\sigma_t$ & 0.085 & 0.060 & 5.692 \\
        \hline $\sigma_p$ & 0.085 & 0.062 & 4.947 \\
        \hline $\Delta$ & 0.001 & 0.002 & 0.745 \\
        \hline
    \end{tabular}
    
\end{itemize}

\newpage
\section{Conclusions}

\begin{itemize}
    \item Практические значения близки к теоретическим.
    
    При увеличениии значения $n$ в 100 раз все оценки распределений уменьшаются, однако, оценка полусуммы минимума и максимума вариационного ряда распределения Лапласа не уменьшается, следовательно, для данного распределения эта оценка не является состоятельной.
    
    \item C точки зрения квадратичного риска лучшая оценка:
    
    для нормального распределения – выборочное среднее,

    для равномерного распределения – полусумма минимума и максимума,

    для распределения Лапласа – выборочная медиана.
\end{itemize}

\end{document}