\documentclass[12pt, a4paper]{article}

\usepackage[utf8]{inputenc}
\usepackage[russian]{babel}
\parindent 0pt
\parskip 8pt
\usepackage{amsmath}
\usepackage{amssymb}
\usepackage{array}
\usepackage{floatrow}
\usepackage{float}
\usepackage[left=2.3cm, right=2.3cm, top=2.7cm, bottom=2.7cm, bindingoffset=0cm]{geometry}
\usepackage{hyperref}
\usepackage{graphicx}
\usepackage{multicol}
\usepackage{listings}
\usepackage{fancyhdr} 
\usepackage{extramarks}
\usepackage[usenames,dvipsnames]{color}
\usepackage{titlesec}
\usepackage{tikz}
\usepackage[T2A]{fontenc} 
\definecolor{grey}{RGB}{128,128,128}

\pagestyle{fancy}
\fancyhf{}
\lhead{Лекция 5}
\chead{Базы данных}
\rhead{\thepage}
\lfoot{by fadyat}
\cfoot{}
\rfoot{22 марта 2022 г.}
\renewcommand\headrulewidth{0.4pt}
\renewcommand\footrulewidth{0.4pt}

\begin{document}
\section{Как происходит join?}


\includegraphics{Screenshot 2022-03-27 at 19.04.11.png}

Необходимо понимать, что соедининение таблиц - дорогая операция

\section{Нормализация и избыточность}

\emph{Нормализация} -- преобразование отношений к виду, отвечающему нормальной форме

\emph{Денормализация} -- искуственная композиция между отношениями для уменьшения операция соединения при запросах

Нормальная форма - характеристика отношения

Пример избыточности:

\begin{center}
    \begin{tabular}{|c|c|c|c|c|}
         \hline
         \multicolumn{5}{|c|}{Студент} \\
         \hline ФИО & Группа & ОП & Факультет & Форма обучения \\
         \hline
    \end{tabular}
\end{center}

Группа - избыточное дублирование

Программа - избыточное дублирование (узнаваема из номера группы)

$\rightarrow$ траты памяти

\section{Аномалии}

\begin{itemize}
    \item \emph{аномалия-модификация} - изменение значения одной записи повлечет за собой просмотр всей таблицы и изменение некоторых других записей
    \item \emph{аномалия-удаления} - при удалении записи может пропасть и другая информация
    \item \emph{аномалия-добавления} - информацию в таблицу нельзя поместить пока она неполная или требуется дополнительный просмотр таблицы
\end{itemize}

Аномалии приводят к нарушению целостности данных

\section{Функциональные зависимости между атрибутами}

\textbf{Функциональная зависимость}

$X \rightarrow Y$ - в отношении $R$ атрибут $Y$ функционально зависит от атрибута $X$ тогда и только тогда, когда каждому значению атрибута $X$ соответствует в точности одно значение атрибута $Y$

\subsection{Частичная функциональная зависимость}

\emph{Частичная функциональная зависимость} -- зависимость неключевого атрибута от части составного потенциального ключа

{ФИО + Группа} $\rightarrow$ ОП

\subsection{Полная функциональная зависимость}

\emph{Полная функциональная зависимость} -- неключевой атрибут зависит от всего составного ключа

{ФИО + Группа} $\rightarrow$ Форма обучения

\subsection{Транзитивная функциональная зависимость}

\emph{Транзитивная функциональная зависимость} -- существует функциональная зависимость из X в Z, если существует такое множетсво атрибутов Y такое, что есть ФЗ из X в Y и Y в Z

$X \rightarrow Z, \exists Y: X \rightarrow Y, Y \rightarrow Z$

\newpage

\section{Нормальные формы}

\subsection{Первая НФ}

\emph{Первая нормальная форма} - отношение находится в ПНФ, если всего его атрибуты являются простыми

\begin{center}
    \begin{tabular}{|c|c|c|c|c|c|c|}
         \hline \multicolumn{7}{|c|}{Студент}\\
         \hline Ф & И & О & Группа & ОП & Факультет & Форма обучения\\
         \hline
    \end{tabular}
\end{center}

\subsection{Вторая НФ}

\emph{Вторая нормальная форма} -- отношение находится во ВНФ, если оно находится в ПНФ и каждый неключевой атрибут функционально-полно зависит от первичного ключа

\begin{center}
    \begin{tabular}{|c|c|c|c|c|}
         \hline \multicolumn{5}{|c|}{Студент}\\
         \hline Ф & И & О & Группа & Форма обучения\\
         \hline
    \end{tabular}
    \begin{tabular}{|c|c|c|}
         \hline \multicolumn{3}{|c|}{Группа} \\
         \hline Группа & ОП & Факультет \\
         \hline
    \end{tabular}
\end{center}

Группа - внешний ключ

\subsection{Третья НФ}

\emph{Третья нормальная форма} -- отношение находится в ТНФ, если оно находится во ВНФ и все неключевые атрибуты взаимнонезависимы и полностью зависят от первичного ключа

\emph{ТНФ} -- отношение находится в ТНФ, если оно находится во ВНФ и ни один неключевой атрибут не находится в транзитивной функциональной зависимости от первичного ключа

\begin{center}
    \begin{tabular}{|c|c|c|c|c|}
         \hline \multicolumn{5}{|c|}{Студент}\\
         \hline Ф & И & О & Группа & Форма обучения\\
         \hline
    \end{tabular}
    \begin{tabular}{|c|c|}
         \hline \multicolumn{2}{|c|}{Группа} \\
         \hline Группа & ОП \\
         \hline
    \end{tabular}
    \begin{tabular}{|c|c|}
         \hline \multicolumn{2}{|c|}{ОП} \\
         \hline Факультет & ОП \\
         \hline
    \end{tabular}
\end{center}

ОП - внешний ключ

\subsection{Нормальная форма Бойса-Кодда}

\emph{Нормальная форма Бойса-Кодда} - отношение находится в НФ Бойса-Кодда, если детерминанты(зависимые части) всех зависимостей являются потенциальными ключами.

\begin{center}
    \begin{tabular}{|c|c|c|c|}
         \hline \multicolumn{4}{|c|}{Проекты} \\
         \hline Номер студента & ФИО & Номер проекта & Роль \\
         \hline
    \end{tabular}
\end{center}

Считаем ФИО неделимым и уникальным \\
Возникает вопрос что использовать в качестве потенциального ключа? \\
Номер студента + номер проекта // ФИО + номер проекта \\
Соответствует ТНФ\\
Не соответствует НФБК \\

Приведенная к НФБК:

\begin{center}
    \begin{tabular}{|c|c|}
         \hline \multicolumn{2}{|c|}{Студент} \\
         \hline Номер студента & ФИО \\
         \hline
    \end{tabular}
    \begin{tabular}{|c|c|c|}
         \hline \multicolumn{3}{|c|}{Проекты} \\
         \hline Номер студента & Номер проекта & Роль \\
         \hline
    \end{tabular}
\end{center}

\subsection{Четвертая НФ}

\emph{Четвертая нормальная форма} -- отношение находится в ЧНФ, если оно находится в НФБК и не содержит многозначных зависимостей


    \begin{tabular}{|c|c|c|}
         \hline \multicolumn{3}{|c|}{Дисциплина}\\
         \hline Дисциплина & Лектор & Практик \\
         \hline
    \end{tabular}
    $\rightarrow$
    \begin{tabular}{|c|c|}
         \hline \multicolumn{2}{|c|}{Лекторы}\\
         \hline Дисциплина & Лектор \\
         \hline
    \end{tabular}
    \begin{tabular}{|c|c|}
         \hline \multicolumn{2}{|c|}{Практики}\\
         \hline Дисциплина & Практик \\
         \hline
    \end{tabular}

Вопрос приведения к ЧНФ - дискуссионный

\subsection{НФ больших порядков}

Экзотика

\end{document}
