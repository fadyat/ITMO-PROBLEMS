\documentclass[12pt, a4paper]{article}

\usepackage[utf8]{inputenc}
\usepackage[russian]{babel}
\parindent 0pt
\parskip 8pt
\usepackage{amsmath}
\usepackage{amssymb}
\usepackage{array}
\usepackage{floatrow}
\usepackage{float}
\usepackage[left=2.3cm, right=2.3cm, top=2.7cm, bottom=2.7cm, bindingoffset=0cm]{geometry}
\usepackage{hyperref}
\usepackage{graphicx}
\usepackage{multicol}
\usepackage{listings}
\usepackage{fancyhdr} 
\usepackage{extramarks}
\usepackage[usenames,dvipsnames]{color}
\usepackage{titlesec}
\usepackage{tikz}
\usepackage[T2A]{fontenc} 
\definecolor{grey}{RGB}{128,128,128}

\pagestyle{fancy}
\fancyhf{}
\lhead{Лекция 5}
\chead{Базы данных}
\rhead{\thepage}
\lfoot{by fadyat}
\cfoot{}
\rfoot{15 марта 2022 г.}
\renewcommand\headrulewidth{0.4pt}
\renewcommand\footrulewidth{0.4pt}


\begin{document}
\section{Базис реляционной алгебры}

Преимущества реляционной модели данных: сохранение целостности данных

\begin{itemize}
    \item Ввести объект
    \item Определить операции
    \item Законы
\end{itemize}

Объект - отношение

\section{Операции}

\subsection{Проекция}

$P_{a_i, a_{i + 1}, ..., a_n} (R)$ -- операция, которая определяет новое отношение содержащие вертикальное подмножество исходного отношения, создаваемое посредством извлечения значений указанных атриубтов и исключения из результата строк дубликатов

\subsection{Выборка}

$\sigma_{предикат} (R)$ -- операция, которая определяет результирующее отношение, которое содержит только те кортежи из исходного отношения, которые удовлетворяют заданному условию(предикату)

\subsection{Объединение}

$R \cup S$ -- определяет новое отношение, которое включает все кортежи содержащиеся только в $R$, которое включает все кортежи содержащиеся только в $S$, и кортежи содержащиеся в $R$ и $S$ без дубликатов.

Совместимость по объеденинению -- два отношения $R$ и $S$ совместимы, когда они состоят из одинакового числа атрибутов и каждая пара соответствующих атрибутов будет иметь одинаковый домен

\subsection{Разность}

 $R - S$ -- отношение, которое состоит из кортежей, которые есть в отношении $R$, но отсутствуют в отношении $S$ требуётся совместимость по объединению
 
\subsection{Пересечение}

$R \cap S$ -- отношение, которое содержит кортежи которые есть в отношении $R$ и в отношении $S$, требуется совместимость по объединению

\subsection{Декартово произведение}

$R \times S$ -- определяет новое отношение, которое является результатом какой-то комбинации каждого кортежа из отношения $R$ с каждым кортежем из отношения $S$.

\subsection{Тета соединение}

$R \bowtie_F S$, $F = R_{a_i} \theta S_{b_i}$, $\theta \in [\leqslant, \geqslant, <, >]$ -- определяет новое отношение, которое содержит кортежи из декартового произведение $R$ и $S$ удовлетворяющие предикату $F$

Если тета в предикате -=, то такое соединение называется экви-соединение

\subsection{Естественное соединение}

$R \bowtie_F S$ -- соединие по эквивалентности двух отношений, выполененное по всем общим атрибутам, из результатов которого исключается по одному экземпляру каждого общего атрибута

\subsection{Левое внешнее соединение}

$R \supset \triangleleft S$ -- тета соединение, при котором результирующее отношение включаются так же кортежи отношения $R$ не имеющих совпадающих значений в общих столбцах отношения $S$

\subsection{Полусоединение}

$R \triangleright_F S$ -- те кортежи из $R$ которые входят в соединение $R$ и $S$

\newpage
\section{SQL}

Не процедурный язык

\includegraphics{tmp.png}

\end{document}
