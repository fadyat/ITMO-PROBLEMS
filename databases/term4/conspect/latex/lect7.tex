\documentclass[12pt, a4paper]{article}

\usepackage[utf8]{inputenc}
\usepackage[russian]{babel}
\parindent 0pt
\parskip 8pt
\usepackage{amsmath}
\usepackage{amssymb}
\usepackage{array}
\usepackage{floatrow}
\usepackage{float}
\usepackage[left=2.3cm, right=2.3cm, top=2.7cm, bottom=2.7cm, bindingoffset=0cm]{geometry}
\usepackage{hyperref}
\usepackage{graphicx}
\usepackage{multicol}
\usepackage{listings}
\usepackage{fancyhdr} 
\usepackage{extramarks}
\usepackage[usenames,dvipsnames]{color}
\usepackage{titlesec}
\usepackage{tikz}
\usepackage[T2A]{fontenc} 
\definecolor{grey}{RGB}{128,128,128}

\pagestyle{fancy}
\fancyhf{}
\lhead{Лекция 7}
\chead{Базы данных}
\rhead{\thepage}
\lfoot{by fadyat}
\cfoot{}
\rfoot{5 апреля 2022.}
\renewcommand\headrulewidth{0.4pt}
\renewcommand\footrulewidth{0.4pt}

\begin{document}
\section{Производительность}
Когда и в какой ситуации что выбирать?

Случаи в которых возникают проблемы с производительностью:
\begin{itemize}
    \item Соединение таблиц
    \item Поиск данных, операции со строками
\end{itemize}

\subsection{Механизм индексов}

\begin{center}
    \begin{tabular}{|c|c|c|}
         \hline ISU & ФИО & Группа \\
         \hline 
    \end{tabular}
    \begin{tabular}{|c|c|}
         \hline Группа & ОП \\
         \hline
    \end{tabular}
\end{center}

\emph{Индекс} -- метод, который позволяет получить структуру данных, предназначенную для оптимизации поиска

Существует вне хранимого отношения, как отдельная структура данных

\begin{center}
    \begin{tabular}{|c|c|}
         \hline key & value - address of tuple \\
         \hline
    \end{tabular}
\end{center}


\subsubsection{Виды индексов}
\begin{itemize}
    \item \emph{Первичный индекс} -- файл упорядочен по первичному ключу и по нему же построен индекс, так что гарантируется уникальность каждого кортежа
    
    map<int> $\rightarrow$ T
    
    \item \emph{Кластерный индекс} -- файл упорядочен по ключевому или по неключевому атрибуту, по которому постороен индекс, при этом несколько кортежей(кластер) соответствуют одному значению индекса
    
    map<int> $\rightarrow$ vector<T>
    
    \item \emph{Вторичный индекс} -- построен по атрибуту отличного от того, по которому осуществлено упорядочивание 
\end{itemize}
Может быть любое число вторичных индексов, но только один первичный и кластерный

\subsubsection{Проблемы индексов}
\begin{itemize}
    \item Перестройка индексов
    \item Рост индексов
\end{itemize}

\subsubsection{Виды индексов}
\begin{itemize}
    \item \emph{Плотный индекс} -- индекс, охватывающий все записи
    \item \emph{Разреженный индекс} -- индекс, охватывающий только определенные блоки
\end{itemize}

\subsection{Механизм представлений}
\emph{Представление} -- динамически сформированный результат одной или нескольких реляционных операций, выполненных над отношениями с целью получения нового отношения

\begin{center}
    \begin{tabular}{|c|c|c|c|}
         \hline ISU & ФИО & Группа & Паспорт \\
         \hline 
    \end{tabular}
    \begin{tabular}{|c|c|}
         \hline Группа & ОП \\
         \hline
    \end{tabular}
    \begin{tabular}{|c|c|}
         \hline ОП & Факультет \\
         \hline
    \end{tabular}
    
    $\big\Downarrow$
    
    \begin{tabular}{|c|c|c|c|c|}
        \hline ISU & ФИО & Группа & ОП & Факультет \\
        \hline
    \end{tabular}
\end{center}

\subsubsection{Виды представлений}

\begin{itemize}
    \item \emph{Материализованное представление} -- непосредственно в памяти храним дубликаты данных, полученные в результате выполнения некоторого запроса, его актуально поддерживается за счет специальной процедуры
    \item \emph{Представление замены} -- храним структуру подзапроса
\end{itemize}
Представление замены решает больше задачу безопасносности нежели производительности

\subsubsection{Цели построения представлений}
\begin{itemize}
    \item Независимость от данных - внешние приложения буду удачно работать при редактировании мной исходной базы данных
    \item Повышение защищенности данных
    \item Снижение сложности запросов
\end{itemize}

\subsubsection{Недостатки построения представлений}
\begin{itemize}
    \item Ограниченные возможности обновлений
    
    \emph{Обновляемое представление} -- могу внести изменение в само представление и существует надежный способ, позволяющий отобразить эти изменения на таблице
    \item Структурные ограничения - если я что-то меняю в моих таблицах нужно не забыть внести изменения в представления - проблемы с целостностью данных
    \item Снижение производительности 
\end{itemize}
\end{document}
