\documentclass[12pt, a4paper]{article}

\usepackage[utf8]{inputenc}
\usepackage[russian]{babel}
\parindent 0pt
\parskip 8pt
\usepackage{amsmath}
\usepackage{amssymb}
\usepackage{array}
\usepackage{floatrow}
\usepackage{float}
\usepackage[left=2.3cm, right=2.3cm, top=2.7cm, bottom=2.7cm, bindingoffset=0cm]{geometry}
\usepackage{hyperref}
\usepackage{graphicx}
\usepackage{multicol}
\usepackage{listings}
\usepackage{fancyhdr} 
\usepackage{extramarks}
\usepackage[usenames,dvipsnames]{color}
\usepackage{titlesec}
\usepackage{tikz}
\usepackage[T2A]{fontenc} 
\definecolor{grey}{RGB}{128,128,128}

\pagestyle{fancy}
\fancyhf{}
\lhead{Лекция 9}
\chead{Базы данных}
\rhead{\thepage}
\lfoot{by fadyat}
\cfoot{}
\rfoot{19 апреля 2022.}
\renewcommand\headrulewidth{0.4pt}
\renewcommand\footrulewidth{0.4pt}

\begin{document}


\section{Безопасность}

Корреляция - связанность между какими-либо признаками, при изменении одного признака - изменяются другие.

\emph{Безопасная компьютерная система} -- по средствам специальных механизмов защиты контролируется доступ к информации, таким образом, что только имеющие соответствующие полномочия лица или процессы, выполняющиеся от их имени, могут получить доступ на чтение, изменение, создание или удален


Аудит любой системы и найти к какому уровню безопасности соответствует 

\section{Уровни безопасности}

CRUD \\ 
|\ D \\
|\ C \\
|\ B \\
|\ A \\

\subsection{D}

Система идентификации и аутентификации, подсистема подсчета событий, связанных с безопасностью и избирательный(дискреционный) контроль доступа

\emph{Идентификация} -- присвоение некоторых идентификаторов субъекту

\begin{itemize}
    \item То что субъекту ... (пароль)
    \item То что субъекту принадлежит (номер телефона)
    \item То что является характеристикой субъекта (биометрия)
\end{itemize}

\emph{Аутентифакиция} -- сопоставление ...

\emph{Авторизация} -- назначение тех или иных прав доступа тому, кто прошел ауте

\subsection{C}
\emph{Дискреционный контроль доступа} -- в том или ином виде существует матрица субъект/объект, на пересечении - CRUD права

\begin{itemize}
    \item Система имеет одного выделенного субъекта и только он имеет право устанавливать любые другие права
    
    \item Каждый объект системы имеет привязанного к себе 'владельца', который может назначать права доступа
    
    \item Субъект с определенным правом доступа может передать данное право другому субъекту
\end{itemize}

ACL - access control list


\subsubsection{C1}
Требование разделения пользователей и данных и определение контура обеспечения безопасности

\emph{Доверенная вычислительная база} -- совокупность защитных механизмов, включающих аппаратное ПО, отвечающих за проведение ... политики безопасности

Должны иметь средства проверки на то что контур безопасности работает корректно

\subsubsection{C2}
Журнал контроля доступа к системе, изоляция ресурсов

\emph{Изоляция ресурсов} -- при выделении объекта из определенного пула вычислительной базы, затем удаляются следы его использования

Проводится тестирование механизма ресурсов


\subsection{B}

Применение мандатного доступа(вместо дискреционного)
\begin{itemize}
    \item Каждому объекту и субъекту ставится в соответствие 'метка секретности'

    \item Субъект может получать доступ на чтение объектов с его уровнем доступа или ниже

    \item Субъект имеет право на запись только в объекты со своим уровнем доступа или выше
\end{itemize}
\subsubsection{B1}

Мандатное управление доступа к выбранным субъктам и объектам

\subsubsection{B2}

Абсолютно любой объект и объект должны быть классифицированны и включены в систему управления мандатным доступом

\subsubsection{B3}

Включает B2 + выделение спецального домена безопасности

Домен характеризуется наличием специального администратора безопасности и системой ...

\subsection{A}

Все функции B3 + формализованные процедуры проектирования и распространения


\section{Ролевая модель}

\emph{Ролевая модель доступа} -- матрица строится относительно роль/субъект

Быстрая проверка по ACL ролей

\section{Про безопасность}

Аудит -- пытаемся записывать все действия и по ним пытаемся вычислить наличие аномалий

Модели доступа помогают решать проблемы штатного доступа

Виды шифрования: 

\begin{itemize}
    \item \emph{Прозрачное шифрование БД} -- пока я работаю с данными в памяти - не шифрованный вид, на диске все данные зашифрованы
    
    Ключ - один на все
    
    \item \emph{Шифрование на уровне столбцов} -- для разных столбцов - разные ключи
    
    Необходимо передавать ключ, нет доступа к данным без ключа
    
    \item \emph{Шифрование с тестами файловой системы} -- при записи файловая система шифрует данные, а не БД
    
    \item \emph{Шифрование на уровне приложений} -- данные всегда в зашифрованном виде, шифруем на стороне приложения
    
    Проблемы с производительностью, надежностью
    
\end{itemize}

Всегда должен быть субъект с наивысшими правами доступа

Проблема разрешима через: 
\begin{itemize}
    \item Резервное копирование (через многоуровневые системы)
    \item Создание систем с невозможностью изменения данных
\end{itemize}


\end{document}