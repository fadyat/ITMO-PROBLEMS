\documentclass[12pt, a4paper]{article}

\usepackage[utf8]{inputenc}
\usepackage[russian]{babel}
\parindent 0pt
\parskip 8pt
\usepackage{amsmath}
\usepackage{amssymb}
\usepackage{array}
\usepackage{floatrow}
\usepackage{float}
\usepackage[left=2.3cm, right=2.3cm, top=2.7cm, bottom=2.7cm, bindingoffset=0cm]{geometry}
\usepackage{hyperref}
\usepackage{graphicx}
\usepackage{multicol}
\usepackage{fancyhdr} 
\usepackage{extramarks}
\usepackage[usenames,dvipsnames]{color}
\usepackage{titlesec}
\usepackage{tikz}
\usepackage[T2A]{fontenc} 
\definecolor{grey}{RGB}{128,128,128}

\pagestyle{fancy}
\fancyhf{}
\lhead{Лекция 1}
\chead{Базы данных}
\rhead{\thepage}
\lfoot{by fadyat}
\cfoot{}
\rfoot{8 февраля 2022 г.}
\renewcommand\headrulewidth{0.4pt}
\renewcommand\footrulewidth{0.4pt}


\begin{document}
\section{Информация}
Информация как:
\begin{itemize}
    \item сигнал (не надо знать что именно)
    \item данные (формализованная информация)
    \item знания (применение, создание)
\end{itemize}

\section{Данные}
\emph{Данные} -- поддающиеся многократной интерпретации представления информации в формализованном виде,  пригодные для передачи, интерпретации или обработки.

Данные позволяют строить интерпретации

\subsection{Модель 1}
\begin{center}
    \begin{tabular}{| c | c | c | c |}
        \hline
        Student & Group & Discipline & Teacher \\
        \hline
    \end{tabular}
\end{center}

\begin{itemize}
    \item Много строчек = $Student \cdot Group \cdot Discipline \cdot Teacher$
    \item Трудно обеспечимая целостность данных
    \item Медленный поиск
    \item Данные неуникальны
\end{itemize}

\subsection{Модель 2}
\begin{center}
	\begin{tabular}{| c | c | c | c |}
    	\hline
        Student & Group & Teacher discipline 1 & Teacher discipline 2 \\
        \hline
    \end {tabular}
\end{center}

\begin{itemize}
    \item Количество памяти меньше прошлой, хотя хранят одинаковую информацию
    \item Число строк = Student
    \item Student - уникален, остальные данные неуникальны
    \item Ускорение поиска
    \item Трудно масштабируемая - необходимо менять структуру
\end{itemize}


\subsection{Модель 3}
\begin{center}
    \begin{tabular}{|c|c|c|}
        \hline
         Student & Id & Group\_id\\
        \hline
    \end{tabular}   
    \begin{tabular}{|c|c|c|}
        \hline
         Group & Id & Student\_id \\
        \hline
    \end{tabular}    
\end{center}

\begin{center}
    \begin{tabular}{|c|c|c|}
        \hline
         Teacher & Id & Lesson\_id \\
        \hline
    \end{tabular}
    \begin{tabular}{|c|c|}
         \hline
            Group\_id & Lesson\_id\\
         \hline
    \end{tabular}
\end{center}

\begin{itemize}
    \item Уменьшение памяти
    \item Улучшение ситуации с целостностью данных
    \item Масштабируемая
    \item В момент запроса необходима память для поиска - использование join
\end{itemize}

\section{Проблема баз данных}

Базы данных так же как и ОС чем-то жертвуют для оптимизации какого-то другого свойства

\begin{itemize}
    \item Надежность
    \item Масштабируемость
    \item Безопасность
    \item Производительность
\end{itemize}

Файловая система как способ хранения данных - неудачный выбор. Основное противоречие - чтение и запись.

\emph{Многозвенная архитектура} -- это архитектура, подразумевающая разделение компонентов на функциональные группы

Появляется идея сделать прослойку, которая бы разделяла бизнес-логику и данные.

\emph{СУБД} -- абстрагирование данных, контроль над целостностью и надежностью.

БД != СУБД

\end{document}
