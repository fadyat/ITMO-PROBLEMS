\documentclass[12pt, a4paper]{article}

\usepackage[utf8]{inputenc}
\usepackage[russian]{babel}
\parindent 0pt
\parskip 8pt
\usepackage{amsmath}
\usepackage{amssymb}
\usepackage{array}
\usepackage{floatrow}
\usepackage{float}
\usepackage[left=2.3cm, right=2.3cm, top=2.7cm, bottom=2.7cm, bindingoffset=0cm]{geometry}
\usepackage{hyperref}
\usepackage{graphicx}
\usepackage{multicol}
\usepackage{fancyhdr} 
\usepackage{extramarks}
\usepackage[usenames,dvipsnames]{color}
\usepackage{titlesec}
\usepackage{tikz}
\usepackage[T2A]{fontenc} 
\definecolor{grey}{RGB}{128,128,128}

\pagestyle{fancy}
\fancyhf{}
\lhead{Лекция 4}
\chead{Базы данных}
\rhead{\thepage}
\lfoot{by fadyat}
\cfoot{}
\rfoot{28 февраля 2022 г.}
\renewcommand\headrulewidth{0.4pt}
\renewcommand\footrulewidth{0.4pt}


\begin{document}

\section{Определения понятия 'База данных'}

\subsection{По \emph{Конноли и Беггу}}

\emph{База данных} -- это совместно используемый набор логически связанных данных и описание этих данных, предназначенные для удовлетворения информационных потребностей организации

\subsection{По \emph{Дейту}} 
\emph{База данных} -- набор постояннохранимых данных, используемых прикладными системами какого-либо предприятия

\subsection{По \emph{Хомоненко}}

\emph{База данных} -- совокупность специальным образом организованных данных, хранимых в памяти вычислительной системы и отображающих состояния объектов и их взаимосвязей в рассматриваемой предметной области

\section{Определение понятия 'СУБД'}

\subsection{По \emph{Конноли и Беггу}}

\emph{СУБД} -- ПО c помощью которого пользователи могут определять, создавать и поддерживать БД, а также осуществлять к ней контролируемый доступ

\subsection{По \emph{Дейту}} 

\emph{СУБД} -- комплекс языковых и програмных средств, предзназначенные для создания, ведения и совместного использования БД многими пользователями

\newpage

\section{Реляционная БД}

\begin{itemize}
    \item Основа РМД - отношение relation
    \item Схема отношения - совокупность заголовков столбцов
    \item Кортеж - отдельная строка в таблице
    \item Атрибут - отдельный столбец таблицы
    \item Сущность - отношение в таблице
    \item Поле - пересечение кортежа и атрибута
    \item Домен - множество допустимых значений атрибута
    \item Степень отношения - количество атрибутов
    \item Кардинальность отношения - количество кортежей
\end{itemize}

\section{Свойства кортежей}

\begin{itemize}
\item уникальность имени отношения в реляционной схеме (каждая таблица имеет определенное имя)
\item каждая ячейка содержит только одно неделимое значение (одно из самых тяжелых с точки зрения баталий и принятия решений)
\item уникальность имени атрибута в пределах отношений
\item значение любого атрибута берутся из одного и того же домена (домен определяет значения атрибута в столбце)
\item каждый кортеж уникален
\item порядок следования атрибутов и порядок следования кортежей не имеют значения (важное расхождение с формулировкой Кодда)\footnote{
Все хорошо пока не начинаешь сортировать (выполнять другие операции) данные. \\ Триггеры - поверх БД весится проверка при добавлении / изменении новых данных, ускорение при чтении данных \\}

\end{itemize}

\newpage

\section{Ключи}
Хранение связей осуществляется при помощи ключей.

В реляционных моделях хранятся только отношения все обьекты однотипны.

\begin{itemize}
    \item \emph{Супер-ключ} -- атрибут, или множество атрибутов единственным образом идентифицирующий кортеж (вся схема отношения, совокупность всех атрибутов)

    \item \emph{Потенциальный ключ} -- супер-ключ, который не содержит подмножества, также являющегося супер-ключем, составной потенциальный ключ содержит более одного атрибута

    \item \emph{Первичный ключ} -- один из потенциальных ключей, который выбран для уникальной идентификации кортежей данного отношения

    \item \emph{Внешний ключ} -- атрибут или множество атрибутов, которые соответствуют потенциальному ключу некоторого может быть того же самого отношения
\end{itemize}

\section{Типы связей}

\begin{itemize}
    \item \emph{Один к одному} - первичный ключ к одному из отношений является одновременно и внешним ключём
    
    \begin{center}
        \begin{tabular}{|c|}
             \hline Employee \\
             \hline id\_exp(PK) \\
             \hline
        \end{tabular}
        \begin{tabular}{|c|}
             \hline SalesPerson \\
             \hline id\_sp(PK, FK) \\
             \hline
        \end{tabular}
    \end{center}
    
    \item \emph{Один ко многим} - значение в некотором неключевом поле берутся из значений потенциального ключа другого отношения
    
    \begin{center}
        \begin{tabular}{|c|c|}
             \hline \multicolumn{2}{|c|}{Employee} \\
             \hline id\_exp(PK) & id\_boss(FK)\\
             \hline
        \end{tabular}
        \begin{tabular}{|c|}
             \hline SalesPerson \\
             \hline id\_sp(PK, FK) \\
             \hline
        \end{tabular}
    \end{center}
    
    \item \emph{Многие ко многим} - может быть реализованно только с помощью доп таблицы-связки, в которой содержатся как минимум пары из ключей потенциальных таблиц
    
    \begin{center}
        \begin{tabular}{|c|c|}
             \hline \multicolumn{2}{|c|}{Employee} \\
             \hline id\_exp(PK) & id\_boss(FK)\\
             \hline
        \end{tabular}
        \begin{tabular}{|c|}
             \hline SalesPerson \\
             \hline id\_sp(PK, FK) \\
             \hline
        \end{tabular}
        \begin{tabular}{|c|c|c|}
             \hline \multicolumn{3}{|c|}{SalesPerson\_Product} \\
             \hline id\_pr & id\_sp & id\_reg \\
             \hline
        \end{tabular}
    \end{center}
    \begin{center}
        \begin{tabular}{|c|}
             \hline Product \\
             \hline id\_pr \\
             \hline
        \end{tabular}
        \begin{tabular}{|c|}
             \hline Region \\
             \hline id\_reg \\
             \hline
        \end{tabular}
    \end{center}
\end{itemize}

\section{Целостность}
\begin{itemize}
\item \emph{Сущностная целостность} -- ни один атрибут первичного ключа не может содержать Null значений
\item \emph{Ссылочная целостность} -- если в отношении существует внешний ключ, то его значение должно соответствовать существующему значению потенциального ключа к другом отношении
\end{itemize}
\end{document}
