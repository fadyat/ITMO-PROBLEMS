\documentclass[9pt, a4paper]{article}

\usepackage[utf8]{inputenc}
\usepackage[russian]{babel}
\parindent 0pt
\parskip 8pt
\usepackage{amsmath}
\usepackage{amssymb}
\usepackage{array}
\usepackage{floatrow}
\usepackage{float}
\usepackage[left=2.3cm, right=2.3cm, top=2.7cm, bottom=2.7cm, bindingoffset=0cm]{geometry}
\usepackage{hyperref}
\usepackage{graphicx}
\usepackage{multicol}
\usepackage{listings}
\usepackage{fancyhdr}
\usepackage{extramarks}
\usepackage[usenames,dvipsnames]{color}
\usepackage{titlesec}
\usepackage{tikz}
\usepackage[T2A]{fontenc}
\usepackage{mathtools}
\mathtoolsset{showonlyrefs}
\definecolor{grey}{RGB}{128,128,128}

\pagestyle{fancy}
\fancyhf{}
\chead{Applied Mathematics}
\rhead{\thepage}
\lfoot{by fadyat}
\cfoot{}
\rfoot{09.12.2022}
\renewcommand\headrulewidth{0.4pt}
\renewcommand\footrulewidth{0.4pt}

\begin{document}
    \begin{titlepage}
        \begin{center}
            \vspace*{2cm}
            \Large
            \textbf{Applied Mathematics}

            \vspace{1cm}
            \large
            Laboratory work №2

            \vspace{1cm}
            \textbf{Simplex method for solving linear programming problems}

            \vspace{1cm}
            \textbf{Author:} Artyom Fadeyev, Sergo Elizbarashvili
        \end{center}
    \end{titlepage}
    \newpage


    \section{Матричные игры}\label{sec:matrix_games}

    \textbf{Игра} -- это идеализированная математическая модель колективного поведения
    нескольких лиц, интересы которых различны, что приводит к конфликту.

    \textbf{Теория игр} -- это математическая теория конфликтных ситуаций, в которых
    участники принимают решения, которые влияют на их выигрыш или проигрыш.

    \textbf{Цель теории игр} -- выработка рекомендаций по разумному поведению участников
    конфликта, которые позволят им достичь наибольшего выигрыша (определить оптимальную стратегию).

    \textbf{Ходы:}
    \begin{enumerate}
        \item \textbf{Личные} -- разумное поведение участника игры, которое позволяет ему достичь наибольшего выигрыша.
        \item \textbf{Случайные} -- рандомные действия, которые принимает каждый игрок в отдельности.
    \end{enumerate}

    \textbf{Стратегия} -- совокупность правил, по которым игрок будет действовать при каждом личном ходе,
    в зависимости от ситуации в игре.

    \textbf{Оптимальная стратегия} -- стратегия, которая обеспечивает игру максимально возможный средний выигрыш или
    минимально возможный средний проигрыш, независимо от стратегии соперника.

    \textbf{Матричная игра} -- конечная парная игра с нулевой суммой, то есть в ней конечное количество ходов,
    два игрока и выигрыш одного игрока равен проигрышу другого.

    Выигрыш(проигрыш) выражается численно.
    В такой игре целью одного игрока будет максимизировать свой выигрыш, целью другого - минимизировать проигрыш.

    Пусть у игрока $A$ есть $n$ стратегий, а у игрока $B$ есть $m$ стратегий.
    Тогда матрица игры имеет размер $n \times m$.

    \begin{table}[h]
        \centering
        \begin{tabular}{|c|c|c|c|}
            \hline                & \textbf{$B_i$} & \ldots & \textbf{$B_m$} \\
            \hline \textbf{$A_j$} & $a_{ij}$       & \ldots & $a_{im}$       \\
            \hline \vdots         & \vdots         & \ddots & \vdots         \\
            \hline \textbf{$A_n$} & $a_{nj}$       & \ldots & $a_{nm}$       \\
            \hline
        \end{tabular}
        \caption{Матрица игры}
        \label{tab:matrix}
    \end{table}

    Если такая таблица задана, то игра приведена к матричной форме.
    В некоторых заданиях требуется работать с уже приведенной формой.

    \newpage

    \subsection{Maximin-стратегия}\label{subsec:maximin-stategy}

    Первая из возможным стратегий - придерживаться принципа \textbf{максимин-стратегии}:
    первый игрок выбирает стратегию, при которой он получит максимальный из минимальных выигрышей,
    второй выбирает минимальный из максимальных проигрышей.

    Первое значени называется \textbf{нижней ценой игры}, а второе - \textbf{верхней ценой игры}.
    Если они равны, то матрица содержит \textbf{седловую точку}.

    Если каждый из противников применяет одну и ту же стратегию, то игра проходит в
    \textbf{чистой стратегии}.

    Случайная величина, значениями которой являются чистые стратегии, называется
    \textbf{смешанной стратегией} игрока.

    Задание смешанной стратегии состоит в указании тех вероятностей, с которыми
    выбираются его чистые стратегии.

    Частный случай: если все вероятности, кроме одной, равны нулю, то смешанная стратегия
    превращается в чистую.

    Пара стратегий $(A_i, B_j)$ называется \textbf{равновесной}, если никому из игроков
    не выгодно изменить свою стратегию.

    Если первый игрок применяет смешанную стратегию, то средний выигрыш первого игрока
    равен:

    \begin{equation}
        \sum_{i=1}^n \sum_{j=1}^m a_{ij} p_i q_j\label{eq:equation}
    \end{equation}

    Ожидаемый проигрыш второго игрока равен той же величине.

    Если матричная игра не имеет седловой точки, то игрок должен руководствоваться теми
    же принципами максимина.

    Формируя стратегию, первый игрок $A$ при которой он получит максимальный из минимальных выигрышей,
    второй игрок $B$ выбирает минимальный из максимальных проигрышей.

    \begin{equation}
        \max_{i} \min_{j} \sum_{i=1}^n \sum_{j=1}^m a_{ij} p_i q_j = v_A\label{eq:equation2}
    \end{equation}

    \begin{equation}
        \min_{j} \max_{i} \sum_{i=1}^n \sum_{j=1}^m a_{ij} p_i q_j = v_B\label{eq:equation3}
    \end{equation}

    \textbf{Основная теорема матричных игр} -- любая матричная игра имеет, по крайней мере,
    одно оптимальное решение, в общем случае, в смешанных стратегиях и соответствующую
    цену $v$.


    Следовательно, любая матричная игра имеет цену $v$.
    Цена игры $v$ называется -- средним выигрыш, приходящиймся на одну партию $\rightarrow$
    всегда удовлетворяет условию $v_l \le v \le v_r$.

    \subsection{Перевод матричной игры в эквивалентную задачу линейного программирования}\label{subsec:to_linear_programming}

    Предварительно нужно убедиться, что все элементы матрицы положительны.
    Чтобы этого добиться, можно прибавить необходимое число $C$, тогда цена игры увеличится на $C$,
    а оптимальные решения не изменятся.

    Найдем смешанную стратегию игрока $A$.
    Предположим, что игрок B применяет только чистые стратегии.
    В каждом случае выигрыш игрока $A$ будет не меньше чем $v$.

    \begin{equation}
        \begin{cases}
            a_{11}p_1 + a_{12}p_2 + \dots + a_{1m}p_m \ge v\\
            a_{21}p_1 + a_{22}p_2 + \dots + a_{2m}p_m \ge v\\
            \dots\\
            a_{n1}p_1 + a_{n2}p_2 + \dots + a_{nm}p_m \ge v
        \end{cases}\label{eq:equation4}
    \end{equation}

    Пусть $x_i = \frac{p_i}{v}$, тогда

    \begin{equation}
        \begin{cases}
            a_{11}x_1 + a_{12}x_2 + \dots + a_{1m}x_m \ge 1\\
            a_{21}x_1 + a_{22}x_2 + \dots + a_{2m}x_m \ge 1\\
            \dots\\
            a_{n1}x_1 + a_{n2}x_2 + \dots + a_{nm}x_m \ge 1\\
            x_i \ge 0, \forall i
        \end{cases}\label{eq:equation5}
    \end{equation}

    Так как игрок $A$ стремится максимизировать свой выигрыш, то можем
    сформулировать задачу линейного программирования:

    Найти значения $x$, такие что удовлетворяют системе ограничений
    и обращаются в минимум такую целевую функцию:

    \begin{equation}
        L(x) = \sum_{i=1}^m x_i\label{eq:equation6}
    \end{equation}

    Из решения найдем цену игры и оптимальную стратегию игрока $A$
    по формулам:

    \begin{equation}
        v = \frac{1}{\sum_{i=1}^m x_i}\label{eq:equation7}
    \end{equation}

    \begin{equation}
        p_i = x_i\cdot v, \forall i\label{eq:equation8}
    \end{equation}

    Аналогично можно найти оптимальную стратегию игрока $B$, но
    неравенства будут иметь обратный знак, а целевую функцию нужно
    будет максимизировать.
    Также матрица коэффициентов будет транспонированной.
    Такая задача называется \textit{двойственной}.

    \newpage


    \section{Решение задач}\label{sec:solve}

    \subsection{Задача 1}\label{subsec:task1}

    \includegraphics[width=1\textwidth]{docs/1}

    Предположим, что компания $F_1$ -- первый игрок, а $F_2$ -- второй.

    Стратегии каждого из игроков: построить универсам в городе $G_i, i = 1, 2, 3, 4$.

    Составим матрицу, в которой каждый элемент таблицы - количество тысяч человек,
    которые будут ходить в магазин первого игрока,
    если построить магазин первого игрока в городе $G_i$,
    а магазин второго игрока в городе $G_j$.

    Рассмотрим ситуацию, когда первый игрок построит магазин в городе $G_1$,
    а второй игрок в городе $G_2$.

    Посчитаем сколько людей будут ходить в магазин первого игрока:

    \begin{equation}
        30 * 0.75 + 50 * 0.45 + 40 * 0.45 + 30 * 0.45 = 76.5\label{eq:equation9}
        \text{ тыс. человек}
    \end{equation}

    Чтобы посчитать сколько людей будут ходить в магазин второго игрока нужно от 150 (население всех городов)
    отнять полученное значение для первого игрока.

    Аналогично считаем значения для всех остальных ситуаций, получаем следующую платёжную матрицу:
    \begin{table}[h]
        \centering
        \begin{tabular}{|c|c|c|c|c|c|}
            \hline        & $G_1$ & $G_2$ & $G_3$ & $G_4$ & $\min$ \\
            \hline $G_1$  & 90    & 76.4  & 91.5  & 91.5  & 76.5   \\
            \hline $G_2$  & 103.5 & 90    & 91.5  & 103.5 & 90     \\
            \hline $G_3$  & 88.5  & 88.5  & 90    & 103.5 & 88.5   \\
            \hline $G_4$  & 88.5  & 76.5  & 76.5  & 90    & 76.5   \\
            \hline $\max$ & 103.5 & 90    & 91.5  & 103.5 &        \\
            \hline
        \end{tabular}\label{tab:table}
    \end{table}

    Данная игра проходит в чистых стратегиях (противники применяют одну и ту же стратегию), значит она имеет седловую точку.

    Найдем её используя принципы минимакса и максимина:

    \begin{equation}
        \max\min = \max(76.5, 90, 88.5, 76.5) = 90\label{eq:equation10}
    \end{equation}

    \begin{equation}
        \min\max = \min(103.5, 90, 91.5, 103.5) = 90\label{eq:equation11}
    \end{equation}

    Значит седловая точка игры - $(G_2, G_2)$.

    Следовательно обоим игрокам выгодно построить магазин в городе $G_2$.
    \newpage

    \subsection{Задача 2}\label{subsec:task2}
    \includegraphics[width=1\textwidth]{docs/2}

    Пусть $x_{ij}$ -- объем работы, выполненный $i$-м погрузчиком на $j$-м объекте.

    Тогда получаем итоговую стоимость работ двух погрузчиков:

    \begin{equation}
        F = 8x_{11} + 7x_{12} + 12x_{21} + 13x_{22} \rightarrow \min\label{eq:equation12}
    \end{equation}

    При этом имеем слудующие ограничения:

    \begin{equation}
        \begin{cases}
            x_{11} + x_{21} = 230\\
            x_{12} + x_{22} = 68\\
            \frac{x_{11}}{10} + \frac{x_{12}}{12} \leqslant 24\\
            \frac{x_{21}}{13} + \frac{x_{22}}{13} \leqslant 24\\
            \frac{x_{11}}{10} + \frac{x_{21}}{13} \leqslant 24\\
            \frac{x_{12}}{12} + \frac{x_{22}}{13} \leqslant 24\\
        \end{cases}\label{eq:equation13}
        \rightarrow
        \begin{cases}
            x_{11} + x_{21} = 230\\
            x_{12} + x_{22} = 68\\
            6x_{11} + 5x_{12} \leqslant 1440\\
            x_{21} + x_{22} \leqslant 312\\
            13x_{11} + 10x_{21} \leqslant 3120\\
            13x_{12} + 12x_{22} \leqslant 3744\\
        \end{cases}\label{eq:equation14}

        \begin{equation}
            x_{ij} \geqslant 0 \ \forall i, j\label{eq:equation15}
        \end{equation}
    \end{equation}

    После решения задачи симпекс-методом получаем следующие значения:

    \begin{equation}
        \begin{cases}
            x_{11} = \frac{550}{3}\\
            x_{12} = 68\\
            x_{21} = \frac{140}{3}\\
            x_{22} = 0\\
        \end{cases}\label{eq:equation16}
        \newline
        F = \frac{7508}{3}
    \end{equation}

    \newpage

    \subsection{Задача 3}\label{subsec:task3}
    \includegraphics[width=1\textwidth]{docs/3}

    Пусть $x_1$ -- количество кг сена, входящее в суточный рацион,
    $x_2$ -- количество кг силоса соответственно.

    Тогда функция для расчета стоимости корма имеет вид:

    \begin{equation}
        F = 12x_1 + 8x_2 \rightarrow \min\label{eq:equation17}
    \end{equation}

    При этом имеем следующие ограничения:

    \begin{equation}
        \begin{cases}
            40x_1 + 10x_2 \geqslant 1000\\
            1.25x_1 + 2.5x_2 \geqslant 100\\
            2x_1 + x_2 \geqslant 80\\
            x_1 + x_2 \geqslant 60\\
        \end{cases}\label{eq:equation18}
        \rightarrow
        \begin{cases}
            40x_1 + 10x_2 \geqslant 1000\\
            5x_1 + 10x_2 \geqslant 400\\
            2x_1 + x_2 \geqslant 80\\
            x_1 + x_2 \geqslant 60\\
        \end{cases}
    \end{equation}

    После решения задачи симпекс-методом получаем следующие значения:

    \begin{equation}
        \begin{cases}
            x_1 = 20\\
            x_2 = 40\\
        \end{cases}\label{eq:equation19}
        \newline
        F = 560
    \end{equation}

    \newpage

    \subsection{Задача 4}\label{subsec:task4}
    \includegraphics[width=1\textwidth]{docs/4}

    Чтобы получить матрицу выигрышей для первого игрока, необходимо
    транспонировать матрицу его проигрышей.

    Тогда получаем:

    \begin{equation}
        A^T =
        \begin{pmatrix}
            4 & 2 \\
            2 & 3 \\
        \end{pmatrix}\label{eq:equation21}
    \end{equation}

    Используем симплекс-метод:
    \begin{enumerate}
        \item
        \begin{equation}
            F(x) = x_1 + x_2 \rightarrow \max\label{eq:equation22}
        \end{equation}
        \begin{equation}
            \begin{cases}
                4x_1 + 2x_2 \leqslant 1\\
                2x_1 + 3x_2 \leqslant 1\\
            \end{cases}\label{eq:equation23}
        \end{equation}

        \item
        \begin{equation}
            F(x) = x_1 + x_2 \rightarrow \min\label{eq:equation24}
        \end{equation}
        \begin{equation}
            \begin{cases}
                4x_1 + 2x_2 \geqslant 1\\
                2x_1 + 3x_2 \geqslant 1\\
            \end{cases}\label{eq:equation25}
        \end{equation}
    \end{enumerate}

    Получаем:

    \begin{enumerate}
        \item $x_1 = \frac{1}{8}, x_2 = \frac{1}{4}, F(x) = \frac{3}{8}$
        \item $x_1 = \frac{1}{8}, x_2 = \frac{1}{4}, F(x) = -\frac{3}{8}$
    \end{enumerate}

    Откуда цена игры: $v = \frac{1}{\sum_{i=1}^m x_i} = \frac{1}{\frac{1}{8} + \frac{1}{4}} = \frac{8}{3}$

    Оптимальная стратегия первого игрока:
    \begin{enumerate}
        \item $p_i = x_i * v = \frac{1}{8} * \frac{8}{3} = \frac{1}{3}$
        \item $p_i = x_i * v = \frac{1}{4} * \frac{8}{3} = \frac{2}{3}$
    \end{enumerate}

    Второй игрок имеет аналогичную стратегию.

    \newpage

    \subsection{Задача 5}\label{subsec:task5}
    \includegraphics[width=1\textwidth]{docs/5}

    \begin{equation}
        A^T =
        \begin{pmatrix}
            8 & 4 \\
            4 & 8 \\
            6 & 5 \\
        \end{pmatrix}\label{eq:equation26}
    \end{equation}

    \begin{enumerate}
        \item
        \begin{equation}
            F(x) = x_1 + x_2 + x_3 \rightarrow \max\label{eq:equation27}
        \end{equation}
        \begin{equation}
            \begin{cases}
                8x_1 + 4x_2 + 6x_3 \leqslant 1\\
                4x_1 + 8x_2 + 5x_3 \leqslant 1\\
            \end{cases}\label{eq:equation28}
        \end{equation}

        \item
        \begin{equation}
            F(x) = x_1 + x_2 \rightarrow \min\label{eq:equation29}
        \end{equation}
        \begin{equation}
            \begin{cases}
                8x_1 + 4x_2 \geqslant 1\\
                4x_1 + 8x_2 \geqslant 1\\
                6x_1 + 5x_2 \geqslant 1\\
            \end{cases}\label{eq:equation30}
        \end{equation}
    \end{enumerate}

    Решаем:
    \begin{enumerate}
        \item $x_1 = 0, x_2 = 0.03571429, x_3 = 0.14285714, F(x) = 0.17857143$
        \item $x_1 = 0.10714286, x_2 = 0.07142857, F(x) = -0.17857143$
    \end{enumerate}

    Откуда цена игры:
    \begin{equation}
        v_1 = \frac{1}{\sum_{i=1}^m x_i} = \frac{1}{0.10714286 + 0.07142857} = 5.6\label{eq:equation31}
    \end{equation}

    \begin{equation}
        v_2 = \frac{1}{\sum_{i=1}^m x_i} = \frac{1}{0 + 0.03571429 + 0.14285714} = 5.6\label{eq:equation32}
    \end{equation}

    Стратегии первого игрока:
    \begin{enumerate}
        \item $p_1 = x_1 * v_1 = 0.10714286 * 5.6 = 0.6$
        \item $p_2 = x_2 * v_1 = 0.07142857 * 5.6 = 0.4$
    \end{enumerate}

    Стратегии второго игрока:
    \begin{enumerate}
        \item $p_1 = x_1 * v_2 = 0 * 5.6 = 0$
        \item $p_2 = x_2 * v_2 = 0.03571429 * 5.6 = 0.2$
        \item $p_3 = x_3 * v_2 = 0.14285714 * 5.6 = 0.8$
    \end{enumerate}

    \newpage

    \subsection{Задача 6}\label{subsec:task6}
    \includegraphics[width=1\textwidth]{docs/6}

    \begin{equation}
        A^T =
        \begin{pmatrix}
            7 & 2 & 5 & 3 \\
            2 & 2 & 3 & 2 \\
            5 & 3 & 4 & 1 \\
            1 & 4 & 4 & 6 \\
        \end{pmatrix}\label{eq:equation33}
    \end{equation}

    \begin{enumerate}
        \item
        \begin{equation}
            F(x) = x_1 + x_2 + x_3 + x_4 \rightarrow \max\label{eq:equation34}
        \end{equation}
        \begin{equation}
            \begin{cases}
                7x_1 + 2x_2 + 5x_3 + 1x_4 \leqslant 1\\
                2x_1 + 2x_2 + 3x_3 + 4x_4 \leqslant 1\\
                5x_1 + 3x_2 + 4x_3 + 4x_4 \leqslant 1\\
                3x_1 + 2x_2 + 1x_3 + 6x_4 \leqslant 1\\
            \end{cases}\label{eq:equation35}
        \end{equation}

        \item
        \begin{equation}
            F(x) = x_1 + x_2 + x_3 + x_4 \rightarrow \min\label{eq:equation36}
        \end{equation}
        \begin{equation}
            \begin{cases}
                7x_1 + 2x_2 + 5x_3 + 3x_4 \geqslant 1\\
                2x_1 + 2x_2 + 3x_3 + 2x_4 \geqslant 1\\
                5x_1 + 3x_2 + 4x_3 + 1x_4 \geqslant 1\\
                3x_1 + 4x_2 + 4x_3 + 6x_4 \geqslant 1\\
            \end{cases}\label{eq:equation37}
        \end{equation}
    \end{enumerate}

    Решаем:
    \begin{enumerate}
        \item $x_1 = 0, x_2 = 0.33333333, x_3 = 0, x_4 = 0, F(x) = 0.33333333$
        \item $x_1 = 0, x_2 = 0, x_3 = 0.33333333, x_4 = 0, F(x) = -0.33333333$
    \end{enumerate}

    Откуда цена игры:
    \begin{equation}
        v_1 = \frac{1}{\sum_{i=1}^m x_i} = \frac{1}{0 + 0.33333333 + 0 + 0} = 3\label{eq:equation38}
    \end{equation}

    \begin{equation}
        v_2 = \frac{1}{\sum_{i=1}^m x_i} = \frac{1}{0 + 0 + 0.33333333 + 0} = 3\label{eq:equation39}
    \end{equation}

    Стратегии первого игрока:
    \begin{enumerate}
        \item $p_1 = x_1 * v_1 = 0 * 3 = 0$
        \item $p_2 = x_2 * v_1 = 0.33333333 * 3 = 1$
        \item $p_3 = x_3 * v_1 = 0 * 3 = 0$
        \item $p_4 = x_4 * v_1 = 0 * 3 = 0$
    \end{enumerate}

    Стратегии второго игрока:
    \begin{enumerate}
        \item $p_1 = x_1 * v_2 = 0 * 3 = 0$
        \item $p_2 = x_2 * v_2 = 0 * 3 = 0$
        \item $p_3 = x_3 * v_2 = 0.33333333 * 3 = 1$
        \item $p_4 = x_4 * v_2 = 0 * 3 = 0$
    \end{enumerate}

    \newpage

    \subsection{Задача 7}\label{subsec:task7}
    \includegraphics[width=1\textwidth]{docs/7_1}
    \includegraphics[width=1\textwidth]{docs/7_2}

    Подставив вероятности стратегий для каждого игрока, получим следующие значения:

    \begin{table}[h]
        \begin{tabular}{|c|c|c|c|}
            \hline
            & $\frac{6}{13}$ & $\frac{4}{13}$ & $\frac{3}{13}$ \\
            \hline
            $\frac{6}{13}$ & 1              & -1             & -1             \\
            \hline
            $\frac{3}{13}$ & -1             & -1             & 3              \\
            \hline
            $\frac{4}{13}$ & -1             & 2              & -1             \\
            \hline
        \end{tabular}\label{tab:table2}
    \end{table}

    Используем формулу для получения математического ожидания для первого игрока:

    \begin{equation}
        \sum_{i=1}^n \sum_{j=1}^m a_{ij} p_i q_j =
        \frac{36 - 24 - 18 - 18 - 12 + 27 - 24 + 32 - 12}{169} =
        -\frac{13}{169} \approx -0.0769\label{eq:equation20}
    \end{equation}

    \newpage

    \subsection{Задача 8}\label{subsec:task8}
    \includegraphics[width=1\textwidth]{docs/8}

    Решим исходную систему уравнений:

    \begin{cases}
        x_1 = 0.12\\
        x_2 = 0.08
    \end{cases}

    Представив $x_i = \frac{p_i}{v}$, получим следующие значения:

    \begin{cases}
        7p_1 + 2p_2 \geqslant u\\
        p_1 + 11p_2 \geqslant u\\
        p_1 + p_2 = 1
    \end{cases}

    \begin{equation}
        p_1 + p_2 = x_1 * u + x_2 * u = u * L(x) = 1 \Rightarrow u = 5\label{eq:equation40}
    \end{equation}


    Считаем оптимальные статегии:

    \begin{cases}
        p_1 = 0.6\\
        p_2 = 0.4
    \end{cases}

    \begin{equation}
        A^T =
        \begin{pmatrix}
            7 & 1  \\
            2 & 11 \\
        \end{pmatrix}\label{eq:equation41}
    \end{equation}

    Решим полученную систему:
    \begin{cases}
        x_1 = 0.133\\
        x_2 = 0.066\\
    \end{cases}

    Представив $x_i = \frac{p_i}{v}$, получим следующие значения:

    \begin{cases}
        7p_1 + 1p_2 \leqslant u\\
        2p_1 + 11p_2 \leqslant u\\
        p_1 + p_2 = 1
    \end{cases}

    Считаем оптимальные статегии:

    \begin{cases}
        p_1 = 0.666\\
        p_2 = 0.333
    \end{cases}

    \newpage

    \subsection{Задача 9}\label{subsec:task9}
    \includegraphics[width=1\textwidth]{docs/9}

    Будем считать $A_i$ -- стратегией первого игрока, $B_j$ -- стратегией второго игрока.

    \begin{table}[h]
        \centering
        \begin{tabular}{|c|c|c|c|c|}
            \hline       & $B_1$ & $B_2$ & $B_3$ & $B_4$ \\
            \hline $A_1$ & 7     & 2     & 5     & 1     \\
            \hline $A_2$ & 2     & 2     & 3     & 4     \\
            \hline $A_3$ & 5     & 3     & 4     & 4     \\
            \hline $A_4$ & 3     & 2     & 1     & 6     \\
            \hline
        \end{tabular}\label{tab:table3}
    \end{table}

    Данную матричную игру можно упростить, так как первый игрок всегда предпочтет
    статегию $A_3$ стратегии $A_2$, ибо выигрыши от стратегии $A_2$ меньше, чем от
    стратегии $A_3$.

    Аналогично избавляемся от столбца $B_1$.

    \begin{table}[h]
        \centering
        \begin{tabular}{|c|c|c|c|}
            \hline       & $B_2$ & $B_3$ & $B_4$ \\
            \hline $A_1$ & 2     & 5     & 1     \\
            \hline $A_2$ & 2     & 3     & 4     \\
            \hline $A_3$ & 3     & 4     & 4     \\
            \hline $A_4$ & 2     & 1     & 6     \\
            \hline
        \end{tabular}\label{tab:table5}
    \end{table}

    Методом максимина найдем нижнее и верхнее цены игры.

    \begin{table}[h]
        \centering
        \begin{tabular}{|c|c|c|c|\min|}
            \hline        & $B_2$ & $B_3$ & $B_4$ & $\min$ \\
            \hline $A_1$  & 2     & 5     & 1     & 1      \\
            \hline $A_3$  & 3     & 4     & 4     & 3      \\
            \hline $A_4$  & 2     & 1     & 6     & 1      \\
            \hline $\max$ & 3     & 5     & 6     &        \\
            \hline
        \end{tabular}\label{tab:table6}
    \end{table}

    Находим седловую точку: $(B_2, A_3)$

    Оптимальные стратегии первого игрока: $A_3$, второго игрока: $B_2$.


    \newpage

\end{document}