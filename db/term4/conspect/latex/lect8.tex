\documentclass[12pt, a4paper]{article}

\usepackage[utf8]{inputenc}
\usepackage[russian]{babel}
\parindent 0pt
\parskip 8pt
\usepackage{amsmath}
\usepackage{amssymb}
\usepackage{array}
\usepackage{floatrow}
\usepackage{float}
\usepackage[left=2.3cm, right=2.3cm, top=2.7cm, bottom=2.7cm, bindingoffset=0cm]{geometry}
\usepackage{hyperref}
\usepackage{graphicx}
\usepackage{multicol}
\usepackage{listings}
\usepackage{fancyhdr} 
\usepackage{extramarks}
\usepackage[usenames,dvipsnames]{color}
\usepackage{titlesec}
\usepackage{tikz}
\usepackage[T2A]{fontenc} 
\definecolor{grey}{RGB}{128,128,128}

\pagestyle{fancy}
\fancyhf{}
\lhead{Лекция 8}
\chead{Базы данных}
\rhead{\thepage}
\lfoot{by fadyat}
\cfoot{}
\rfoot{12 апреля 2022.}
\renewcommand\headrulewidth{0.4pt}
\renewcommand\footrulewidth{0.4pt}

\begin{document}
\section{Надежность}
У БД рассматриваем, как минимум два вида надежности:

\begin{itemize}
    \item \emph{Надежность хранения} -- через какое-то время при попытке доступа к данным, данные сохраняют свой первоначальный вид в котором их оставили
    \item \emph{Надежность доступа} -- должен обеспечиваться доступ к данным
\end{itemize}

Целостность, сохранность данных, физическая целостность и сохранность данных

>> Немного про RAID-массивы, СХД, ЦОД, Data center TIER
\subsection{Модель транзакций}

\emph{Транзакция} -- последовательность действий с БД, в которой все действия выполяются успешно, либо не выполняется ни одно из них
 
Результат: commit / rollback

Обеспечение всех свойств обеспечивает логическую целостность данных
\subsubsection{Свойства транзакций}
\begin{itemize}
    \item \emph{Атомарность} -- транзакция неделима, либо выполняется всё, либо ничего
    \item \emph{Согласованность} -- перевод из одного состояния согласованности в другое без соблюдения состояния согласованности в промежуточных точках
    \item \emph{Изоляция} -- если запущены несколько конкурирующих транзакций, то любое обновление состояния БД, выполенное одной транзакцией, скрыто от других до ее завершения
    \item \emph{Долговечность} -- когда траназакция завершена, её результаты сохраняются, даже если в следующие моменты произойдет сбой
\end{itemize}
\subsubsection{Проблемы конкурирующих транзакций}
\begin{itemize}
    \item \emph{Проблема потерянного обновления} -- несколько транзакций меняют один и тот же кортеж, в результате сохранится только изменения внесенные последней транзакцией
    
    \includegraphics[scale=0.5]{pictures/d1.png}
    
    \item \emph{Проблема грязного чтения} -- при чтении одной транзакции кортежа, который уже изменен, но еще не сохранен еще не завершившейся транзакцией, которая потом будет отменена
    
    \includegraphics[scale=0.5]{pictures/d2.png}
    \item \emph{Проблема неповторяемого чтения} -- при повторном чтении данных, уже считанных данных, транзакция обнаруживает модификацию, вызванную другой завершенной транзакцией
    
    \item \emph{Проблема фантомного чтения}

\end{itemize}

Блокировки:
\begin{itemize}
    \item ...
    \item ...
\end{itemize}

\begin{itemize}
    \item Накладывающииеся из приложения (явные)
    \item Накладывающиеся из СУБД (неявные)
\end{itemize}

\begin{itemize}
    \item Монопольные -- блокировка всех видов доступа к объекту
    \item Коллективные -- блокировка на чтение
\end{itemize}

\newpage

\subsubsection{Уровни изоляций}

\begin{itemize}
    \item \emph{Незавершенное чтение} -- требует, чтобы изменять данные могла только одна транзакция
    
    \item \emph{Завершенное чтение} -- если транзакция начала изменение данных, то никакая другая не сможет их прочитать до завершения первой
    
    \item \emph{Воспроизводимое чтение} -- если транзакция считывает данные, то никакая другая не сможет их изменить до завершения первой
    
    \item \emph{Сериализуемость} -- если транзакция обращается к данным, то никакая другая транзакция не сможет добавить или изменить существующие кортежи в этом объекте данных
    
\end{itemize}

Каждый уровень изоляции помогает преодолеть новую проблему конкурирующих транзакций

Долговечность обеспечивается журналированием транзакции
\end{document}