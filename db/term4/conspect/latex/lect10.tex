\documentclass[12pt, a4paper]{article}

\usepackage[utf8]{inputenc}
\usepackage[russian]{babel}
\parindent 0pt
\parskip 8pt
\usepackage{amsmath}
\usepackage{amssymb}
\usepackage{array}
\usepackage{floatrow}
\usepackage{float}
\usepackage[left=2.3cm, right=2.3cm, top=2.7cm, bottom=2.7cm, bindingoffset=0cm]{geometry}
\usepackage{hyperref}
\usepackage{graphicx}
\usepackage{multicol}
\usepackage{listings}
\usepackage{fancyhdr} 
\usepackage{extramarks}
\usepackage[usenames,dvipsnames]{color}
\usepackage{titlesec}
\usepackage{tikz}
\usepackage[T2A]{fontenc} 
\definecolor{grey}{RGB}{128,128,128}

\pagestyle{fancy}
\fancyhf{}
\lhead{Лекция 9}
\chead{Базы данных}
\rhead{\thepage}
\lfoot{by fadyat}
\cfoot{}
\rfoot{26 апреля 2022.}
\renewcommand\headrulewidth{0.4pt}
\renewcommand\footrulewidth{0.4pt}

\begin{document}
\section{Распределенная система БД}

\includegraphics[scale=0.6]{bd1.png}

\subsection{Предпосылки к распределенным БД}

\begin{itemize}
    \item ...
    \item ...
    \item Естественная распределенность данных
\end{itemize}

\emph{Распределенная БД} -- набор, логически связанных между собой разделяемых данных и их описаний, которые физически распределены по нескольким вычислительным узлам


\emph{Фрагментирование} -- разделение одно таблицы на несколько

\subsection{Способы фрагментирования}
\begin{itemize}
    \item Горизонтальное -- выделение подмножеств строк
    
    Когда есть какой-то временной аспект для данных
    
    \item Вертикальное -- пытаемся хранить таблицу, не как кортежи со всеми атрибутами, а как несколько кортежей с какими-то атрибутами
    
    Где удобно задать безопасность для ролевой модели (секретная / несекретная области)
    
    \item Смешанное -- подмножества строк с подмножеством атрибутов
    
    Когда удобно совмещать 
    
\end{itemize}

\emph{Репликация} -- поддержка синхронизированных физических копий некоторого объекта БД

\subsection{Стратегии размещения данных в распределенной системе}

\begin{itemize}
    \item Раздельное (фрагментированное) размещение -- БД разбивается на непересекающиеся фрагменты и каждый фрагмент располагается строго на одном узле
    
    \item Размещение с полной репликацией -- на каждом узле есть полная копия (реплика -- синхронизированная копия) всей БД
    
    Большая стоимость хранения\\
    Высокая надежность\\
    Производительность неоднозначна \\

    \item Размещение с выборочной репликацией -- разделяем БД на фрагменты и для каждого фрагмента пытаемся решить задачи: сколько копий сделать? где копии расположить?
    
\end{itemize}

\subsection{Принципы прозрачности}

\begin{itemize}
    \item Прозрачность фрагментации
    \item Прозрачность расположения
    \item Прозрачность количества реплик
    \item Прозрачность контроля доступа
    
    Пользователь получивший отказ в доступе не знает существуют ли такие данные или у него нет доступа
    
    
\end{itemize}

\subsection{Базы данных}
\begin{itemize}
    \item Гомогенные -- используют одинаковую СУБД на всех узлах
    \item Гетерогенные -- используют разные СУБД на всех узлах
\end{itemize}

\subsection{Правила распределенных БД}

Идеальная 
\begin{itemize}
    \item Локальная автономность -- локальные данные (данные на узле) принадлежат локальным владельцам и локально сопровождаются
    
    \item Отсутствие опоры на центральный узел -- в системе не должно быть ни одного узла, без которого система не может функционировать
    
    \item Непрерывное функционирование -- в системе не должна возникать потребность в плановом останове ее функционирования
    
    \item Независимость от расположения
    
    \item Независимость от фрагментации -- получение доступа к данным
    вне зависимости от их фрагментации
    
    \item Независимость от репликации -- пользователь не должен нуждать в сведениях о наличиях реплик
    
    \item Обработка распределенных запросов -- обработать любой запрос вне зависимости на скольких узлах расположены требуемые объекты данных
    
    \item Обработка распреденных транзакций -- систмема должная поддерживать транзакции с данными, расположенных более чем на одном узле

    \item Независимость от типа оборудования
    
    \item Независимость от сетевой архитектуры
    
    \item Независимость от ОС
    
    \item Независимость от типа СУБД
    
\end{itemize}


\subsection{Задачи обработчика распределенных запросов}

\begin{itemize}
    \item В каком фрагменте расположены нужные мне сведения?
    \item К какой копии фрагмента обращаться?
    \item Где расположить временные структуры?
\end{itemize}

\subsection{Преимущества распределенных систем}
\begin{itemize}
    \item Нативно отображает структуру организации
    \item Разделяемость и локальная автономность
    \item Повышение доступности и надежности
    \item Повышение производительности 
    \item Модульность системы - позволяет распараллелить обработку и обслуживание
\end{itemize}

\subsection{Недостатки распределенных систем}

\begin{itemize}
    \item Повышение сложности на всех уровнях (архитектура, алгоритмы и тп)
    \item Увеличение стоимости владения
    \item Проблема защиты
    \item Усложнение контроля за целостностью данных
    \item Отсутствие стандартов
\end{itemize}

\end{document}