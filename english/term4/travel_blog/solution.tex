\documentclass[12pt, a4paper]{article}

\usepackage[utf8]{inputenc}
\usepackage[russian]{babel}
\parindent 0pt
\parskip 8pt
\usepackage{amsmath}
\usepackage{amssymb}
\usepackage{array}
\usepackage{floatrow}
\usepackage{float}
\usepackage[left=2.3cm, right=2.3cm, top=2.7cm, bottom=2.7cm, bindingoffset=0cm]{geometry}
\usepackage{hyperref}
\usepackage{graphicx}
\usepackage{multicol}
\usepackage{listings}
\usepackage{fancyhdr} 
\usepackage{extramarks}
\usepackage[usenames,dvipsnames]{color}
\usepackage{titlesec}
\usepackage{tikz}
\usepackage[T2A]{fontenc} 
\definecolor{grey}{RGB}{128,128,128}


\begin{document}
\section{Criteria}
\subsection{Vocabulary}
\begin{itemize}
    \item grandeur
    \item breathtaking
    \item campsite
    \item stones pressing into my back
    \item natural phenomenon
    \item to be worth something
    \item rare
    \item the heart of something
    \item pretty amazing
    \item incredible sunrise / sunset
    \item a peak season
    \item see from different viewpoints
    \item re-enactment of an old
    \item lose your mind
    \item madness
    \item \textbf{adjectives from 6D: fabulous, awesome, etc}
    \item \textbf{phrases with experience}
\end{itemize}

\newpage
\subsection{Grammar}
\begin{itemize}
    \item Gerund -- is a verb that's acting as a noun.
    
    \url{https://grammarly.com/blog/gerund} \\
    
    By that, we mean that the verb -- the word that describes the action that's happening, like 'biking', 'thinking', 'running' -- becomes a thing, a concept that can be the sentence's subject, direct object, indirect object, or the object if a preposition. 
    
    It doesn't stop being a verb, but the role it plays in a sentence shifts from describing the action to being a focal point.
    
    Gerund phrases is a phrase that contains a gerund and a modifier or an object and, in some cases, both of these.
    
    \textbf{Types of gerund:} \\
    subject, subject component, direct object, indirect object, object complement, object of a preposition
    
    \item Passive voice
    
    \url{https://www.grammarly.com/blog/passive-voice}
    
    The passive voice is used to show interest in the person or object that experiences an action rather than the person or object that performs the action. In other words, the most important thing or person becomes the subject of the sentence.
\end{itemize}

\newpage

\section{Travel blog}

At some points in my life, a feeling of loneliness succeeds over me. This is not always the case. I have a great time with my family, especially when traveling.

Travel is an integral part of my life. One of the brightest and most memorable is a trip to Berlin. It was multifaceted, including hiking, so boring bus rides, and, of course, excellent German cuisine restaurants. It was a fresh experience. Mistakes were made. All members of our family have formed their descriptions of that city. For me, this is the Berlin Wall, and for my mother, reading at a gorgeous sunrise. This is a truly breathtaking building. I almost lost my head at the coolness of these drawings. Some paintings evoke mixed feelings. They show the courage of the artists and political cruelty.

In addition, I  went up on the roof of the Reichstag building. Once at the top, you have an amazing view of the entire city. If you look down, you can see crowds of people who are in a hurry somewhere, but here - at the top, you are away from all this fuss.

A trip to Berlin is not cheap entertainment, prepare money and your patience. My journey is over. But this is a natural phenomenon in our life. I want to visit Germany again, but in the current situation, it’s difficult. New trip to Germany will be as soon as possible.
\end{document}